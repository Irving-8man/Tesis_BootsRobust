\section*{Resumen}
En este trabajo se propone un método que permite evaluar la precisión de un modelo con la técnica de regresión lineal; y se basa en implementar diversos esquemas de remuestreo y estimar la precisión, a través de intervalos de confianza Bootstrap (ICB) para el coeficiente de determinación $R^2$, del modelo de regresión entre los valores reales y predichos del modelo que se desea evaluar.\

Se consideraron cuatro escenarios posibles (NVC, NNVC, NVD y NNVD), de acuerdo al cumplimiento o no de los supuestos de normalidad y varianza constante; y para la estimación de la distribución del coeficiente de determinación $R^2$, se implementaron ocho esquemas de remuestreo Bootstrap: el simple; los tres propuestos por \textcite{wu-1986}; los dos propuestos por \textcite{liu-1988}; el balanceado y el pareado balanceado. Para la estimación de $R^2$ se propuso el intervalo Bootstrap Percentil y el Bca; y para su cómputo se utilizaron $B=1,000$ remuestras por cada esquema.\

Se realizó un estudio de simulación para comparar las eficiencias de los intervalos de confianza para cada tipo de supuesto con respecto a los diferentes esquemas Bootstrap, tamaños de muestra y tipo de modelo; para ello se simularon y evaluaron $60,000$ modelos Exactos-Precisos (EP) y $60,000$ modelos Exactos-Imprecisos (EI); para cada modelo se identificó la $R^2$ de origen utilizada para su simulación. Para cada uno de los supuestos y tipo de modelo se simularon modelos con tamaños $n=10, 15, 20, 25, 30, 35$.\

Se consideraron tres tipos de eficiencias para los intervalos con cada esquema Bootstrap, la eficiencia como el porcentaje de las veces en que el intervalo contiene a la $R^2$ de origen para los modelos EP simulados; y viceversa, para los modelos EI, se determinó como el porcentaje de las veces en que el intervalo de confianza no contiene a la $R^2$ de origen. También se consideró la eficiencia cuando ambos ICB contienen o no a la $R^2$ de origen y la eficiencia  cuando uno de los intervalos es más estrecho que el otro dado que ambos ICB contienen o no a la $R^2$ de origen.\

Se analizaron los resultados del estudio de simulación a través de un ANOVA factorial y se implementó en el lenguaje R como propuesta final para la evaluación de la precisión de un modelo: cuando los supuestos son NVC, NNVC o NVD se utilice el ICB Percentil con el esquema de remuestreo Liu2 y para el supuesto NNVD se utilice el ICB BCa con el esquema de remuestreo Pareado Balanceado. Finalmente, se aplicó la propuesta en la evaluación de dos modelos que corresponden a casos reales.
