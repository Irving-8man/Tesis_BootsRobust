\section*{Resumen}

En este trabajo se propone un método que permite evaluar la precisión de un modelo con la técnica de regresión lineal; y se basa en implementar diversos esquemas de remuestreos y estimar la precisión, a través de intervalos de confianza Bootstrap para el coeficiente de determinación $R^2$, del modelo de regresión entre los valores reales y predichos del modelo que se desea evaluar.\\

Tomando en cuenta el cumplimiento o no de los supuestos de normalidad y varianza constante, se consideraron cuatro escenarios posibles (NVC, NNVC, NVD y NNVD), como resultados de la combinación de ambos supuestos; y para estimar la distribución del coeficiente de determinación $R^2$, se implementaron ocho esquemas de remuestreos: el Bootstrap simple; el Wild Bootstrap robusto propuesto en \textcite{rana-2012}, ejecutado con los tres esquemas propuestos por \textcite{wu-1986} y con los dos esquemas propuestos por \textcite{liu-1988}; el Bootstrap de residuales balanceado y el Bootstrap pareado balanceado. Se proponen los intervalos percentiles y el Bca para estimar $R^2$ y para su cómputo se utilizan $B=1,000$ remuestras para cada uno de los esquemas Bootstrap.\\ 

Se realizó un estudio de simulación para comparar las eficacias de los intervalos de confianza para cada tipo de supuesto con respecto a los diferentes esquemas Bootstrap, tamaños de muestra y tipo de modelo; para ello se simularon y evaluaron $60,000$ modelos Exactos-Precisos (EP) y $60,000$ modelos Exactos-Imprecisos (EI); para cada modelo se identificó la $R^2$ de origen utilizada para su simulación. Se simularon modelos de tamaños $n=10, 15, 20, 25, 30, 35$ para cada uno de los supuestos y tipo de modelo.\\

Se consideraron tres criterios principales para determinar las eficacias de los intervalos para cada esquema Bootstrap, el primer criterio determina la eficacia como el porcentaje de las veces en que el intervalo contiene a la $R^2$ de origen para los modelos EP simulados; y viceversa, para los modelos EI la eficacia se determinó como el porcentaje de las veces en que el intervalo de confianza no contiene a la $R^2$ de origen. El segundo criterio determina la eficacia como el porcentaje de las veces en que ambos intervalos contienen de manera simultánea a la $R^2$ de origen para los modelos EP y de manera viceversa cuando ambos no la contienen para los modelos EI; y el tercer criterio determina la eficacia como el porcentaje de las veces en que uno de los intervalos es más estrecho que el otro cuando ambos intervalos contienen simultáneamente la $R^2$ de origen para los modelos EP y de manera viceversa uno de los intervalos es más estrecho que el otro cuando ambos no la contienen simultáneamente para los modelos EI.\\

Se analizaron los resultados del estudio de simulación a través de un ANOVA factorial y se determinó con al menos un 95\% que para los supuestos NVC, NNVC o NVD se utilice el ICB Percentil con el esquema de remuestreo Liu2 sin importar el tamaño de muestra. Y se determinó con al menos el 88.8\% que para el supuesto NNVD se utilice el ICB BCa con el esquema de remuestreo pareado balanceado; con la limitación de que para modelos EI con tamaños de muestra “pequeño”$ n=10, 15, 20,$ no se obtuvo un buen desempeño. \\


Con el propósito de que los resultados de este trabajo conformen una herramienta que permita evaluar la precisión de un modelo, se consideró como propuesta final: para los supuestos NVC, NNVC o NVD se utilice el ICB Percentil con el esquema de remuestreo Liu2 y para el supuesto NNVD se utilice el ICB BCa con el esquema de remuestreo pareado balanceado. La propuesta final se implementó en el lenguaje R \parencite{R-2024}; y se ilustra con la evaluación de tres modelos que se ajustan a los diferentes escenarios contemplados en la propuesta y que corresponden a casos reales.