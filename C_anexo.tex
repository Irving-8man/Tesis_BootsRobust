\section*{Anexo C. Programas en R}

\begin{verbatim}
#Librerias
library("robustbase")
library("readxl")
#Función de apoyo para ponderar residuales
#Sea residualesRob <- residuales del modelo con regresión robusta
#CMERob <- cuadrado medio del error del modleo con regresión robusta
PodResidRobu <- function(residualesRob, CMERob){
	consPes <- 3
	n <- length(residualesRob)
	x <- abs(residualesRob)/sqrt(CMERob)
	w <- rep(1,n)
	xx <- which(x > consPes) 
	w[xx] <- (consPes / w[xx])
	resRobuPonder<- w*residualesRob
	return(resRobuPonder)
}
\end{verbatim}


\begin{verbatim}

#Función de apoyo para obtener las muestras Bootstrap de R²
#Sea y <- los observador,
# z <- los estimados,
#yAjRob <- y ajustados 
#residuales <- residuales por regresion lineal,
#residualesRob <- residuales por regresion lineal robusta, 
#residualesRP <- residuales robustos ponderados,
#hii<- valor de aplacamiento del modelo con 
#regresion simple, n <- tamaño de la muestra de residuos, 
#B <- repeticiones Bootstrap,
#tipo <- sea el tipo de esquema Boostrap Robusto. 
#1<-Wu 1, 2<-Wu 2, 3<-Wu 3,
# 4<-Liu 1, 5<-Liu 2, 6<- Robusto/Simple
# 7<-residuales balanceados, 8<- pareado balanceado
CalcularR2Bootstrap <- function(y, z, yAjRob, residuales, 
					residualesRob, residualesRP, hii, n, B, tipo) {
	sqrt_hii <- sqrt(1 - hii)
	RsBoot <- numeric(B)
	
	if((tipo == 7) || (tipo ==8)){
		N     <- rep(1:n,B)
		NPerm <- sample(N)
	}
	
	for (i in 1:B) {
		
		residualBT <- switch(
		tipo,
		{
			tt <- rnorm(n)
			(tt * residualesRP) / sqrt_hii
		},
		{
			ai <- (residuales - mean(residuales)) / sd(residuales)
			tt <- sample(ai, replace = TRUE)
			(tt * residualesRP) / sqrt_hii
		},
		{
			mediana <- median(residualesRP)
			NMAD <- (1 / 0.6745) * median(abs(residualesRP - mediana))
			Rai <- (residualesRP - mediana) / NMAD
			tt <- sample(Rai, replace = TRUE)
			(tt * residualesRP) / sqrt_hii
		},
		{
			tt <- rgamma(n, 2, 4)
			(tt * residualesRP) / sqrt_hii
		},
		{
			media1 <- 0.5 * sqrt(17 / 6) + sqrt(1 / 6)
			media2 <- 0.5 * sqrt(17 / 6) - sqrt(1 / 6)
			H <- rnorm(n, media1, sqrt(0.5))
			D <- rnorm(n, media2, sqrt(0.5))
			tt <- H * D - media1 * media2
			(tt * residualesRP) / sqrt_hii
		},
		{
			sample(residualesRob, replace = TRUE)
		},
		{
			posI <- (i-1)*n+1
			posF <- i*n
			VPosi <- NPerm[posI:posF]
			residualesRP[VPosi]
		},
		{
			posI <- (i-1)*n+1
			posF <-i*n
			NPerm[posI:posF]
		},
		stop("Esquema no válido")
		)
		
		if(tipo != 8){
			yBoots <- yAjRob + residualBT
			modeloBoots <- lm(yBoots ~ z)
		}else{
			YP <- y[residualBT]
			ZP <- z[residualBT]
			modeloBoots <- lm(YP ~ ZP)
		}
		
		RsBoot[i] <- summary(modeloBoots)$r.squared
	}
	
	return(RsBoot)
}
\end{verbatim}


\begin{verbatim}
#Función de apoyo para construir el intervalo de confianza 
#la muestra de R² Bootstraps del modelo
#Sea data <- los valores z e y del modelo,
# R2 <- la R² estimada del modelo, 
#muestrasR2Boot <- la muestra de R² Bootstrap,
# B <- repeticiones Bootstrap, nivConfianza<- nivel de confianza,
#tipo<- sea el tipo de intervalo de confiaza que desea construir,opciones:
#1<- percentil, 2<-BCa

ContruirIntervBoot <- function(data, R2, muestrasR2Boot, B, 
								nivConfianza, tipo){
	
	intervalo <- numeric(2) 
	alpha <- 1-nivConfianza
	z <- as.numeric(data[[1]])
	y <- as.numeric(data[[2]])
	vectorR2Bootstrap <- muestrasR2Boot
	
	intervalo <- switch(
	tipo,
	{
		puntosCriticos <- quantile(vectorR2Bootstrap, 
									c(alpha/2, 1 - alpha/2))
		as.vector(puntosCriticos)
	},
	{
		n <- length(z)
		z0 <- qnorm(mean(vectorR2Bootstrap < R2))
		suma0=0
		suma02=0
		suma03=0
		for(i in 1:n)
		{R2MI=summary(lm(y[-i]~z[-i]))$r.squared
			suma0=R2MI+suma0
		}
		R2PMI=suma0/n
		for(i in 1:n)
		{Dif0=R2PMI-summary(lm(y[-i]~z[-i]))$r.squared
			suma02=Dif0^2+suma02
			suma03=Dif0^3+suma03
		}
		a=suma03/(6*(suma02^1.5))
		
		z_alfa1 <- qnorm(alpha / 2)
		z_alfa2 <- qnorm(1 - alpha / 2)
		alfa1 <- pnorm(z0 + (z0 + z_alfa1) / (1 - a * (z0 + z_alfa1)))
		alfa2 <- pnorm(z0 + (z0 + z_alfa2) / (1 - a * (z0 + z_alfa2)))
		
		ICInfBootBCa <- quantile(muestrasR2Boot, alfa1)
		ICSupBootBCa <- quantile(muestrasR2Boot, alfa2)
		as.vector(c(ICInfBootBCa, ICSupBootBCa))
	},
	stop("Intervalo no válido")
	)
	
	return (intervalo)
}
\end{verbatim}



\begin{verbatim}
	
	#Función propuesta para evaluar la precisión de un modelo creando
	#intervalos de confianza con la tecnica de regresion lineal
	#con estimadores robustos y esquemas Booststrap
	#Sea data<- el modelo con las columnas z e y
	#caso <- 1 #Normalidad- homocedasticidad,
	#2 #Normalidad-heterocidasticidad,
	#3 #No normalidad-homocedasticidad y 
	#4 #No normalidad-heterocidastecidad 
	EvalPrecisionModel <- function (data, alpha, nivConfianza, caso){
		z <- as.numeric(data[[1]])
		y <- as.numeric(data[[2]])
		n <- length(z)
		B <- 1000 #Remuestras bootstrap
		
		#Regresion lineal simple
		modeloLineal <- lm(y~z)
		residuales <- residuals(modeloLineal)
		R2 <- summary(modeloLineal)$r.squared
		hii <- hatvalues(modeloLineal)
		yAju <- fitted(modeloLineal)
		
		#Casos
		NVC <- 1 #Normalidad- homocedasticidad
		NVD <- 2 #Normalidad-heterocidasticidad
		NNVC <- 3 #No normalidad-homocedasticidad
		NNVD <- 4 #No normalidad-heterocidastecidad 
		numRemues <- 8 #Numero de tipos de remuestreos implementados
		rsBoot <- matrix(0,nrow = B, ncol = numRemues)
		
		#Uso de los estimadores dependendiendo el caso
		if(NVC == caso){
			#Minimos cuadrados
			residualesRob <- residuales
			yAjRob <- yAju
			residualesRP <- residuales
			
		}else{
			# Uso de estimador robusto MM  
			modeloLinealRob <- lmrob(y ~ z, method = "MM")
			residualesRob <- modeloLinealRob$residuals
			yAjRob <- modeloLinealRob$fitted.values
			
			#Segundo caso,no cumple normalidad, si varianza
			if(NNVC==caso){
				# Residuales robustos sin ponderacion
				residualesRP <- residualesRob
			}
			
			#Tercer caso, no hay homocedasticidad
			if((NNVD==caso) || (NVD==caso)){
				# Residuales robustos con ponderacion
				CMERob <- modeloLinealRob$scale**2
				residualesRP <- PodResidRobu (residualesRob,CMERob)
			}
		}
		
		#Procesamiento de los residuales en distintos esquemas
		for(i in 1:numRemues){
			BootR <- CalcularR2Bootstrap(y,z, yAjRob,residuales
			,residualesRob,residualesRP,hii,n,B,tipo=i)
			rsBoot[, i] <- BootR
		}
		resultadosInter <- vector("list", numRemues)
		
		#Cálculo de los intervalos para el R²
		for(i in 1:numRemues){
			muestrasR2Boot <- rsBoot[,i]
			perc <- ContruirIntervBoot(data,R2,muestrasR2Boot
			,B,nivConfianza=0.95,tipo=1)
			bca <- ContruirIntervBoot(data,R2,muestrasR2Boot,
			B,nivConfianza=0.95,tipo=2)
			resultadosInter[[i]] <- list(perc, bca)
		}
		
		return(resultadosInter)
	}
\end{verbatim}



\begin{verbatim}
# Función para procesar y capturar los resultados 
#sobre la precisión de las muestras con sus I.C.
# Dado archivos_encontrados <-  list(muestra,R2) 
#con rutas de archivos
# para el parametro caso, 1-NVC, 2-NVD, 3-NNVC, 4-NNVD
# para replicas sea el número de replicas
# con nivConfinza entre 0 - 1
# y sea para N el tamaño de las muestras
ProcesarModels <- function(archivos_encontrados, caso,
						 replicas, nivConfianza, N, MODELO, CASO) {
	library(openxlsx)
	library(readxl)
	archivo_muestra <- archivos_encontrados$muestra
	archivo_R2 <- archivos_encontrados$R2
	data_muestra <- read_excel(archivo_muestra, col_names = TRUE)
	data_R2 <- read_excel(archivo_R2, col_names = TRUE)
	
	block <- 1000
	cols_por_model <- 2
	limit_model <- 500 #modelos a procesar
	esquemas <- 8
	
	#Tablas de conteos
	nombre_cols <- c("Replica","Esquema", "NumMod", 
					"NumModEfic", "FrecEficIB1","FrecEficIB2",
	"FrecEficIB1Unico","FrecEficIB2Unico",
	"FrecEficIB1Emp2", "FrecEficIB2Emp2","NingunGanador")
	conteos_totales <- matrix(0, ncol = length(nombre_cols),
							 nrow = replicas * esquemas)
	colnames(conteos_totales) <- nombre_cols
	
	#Tabla de eficiencia
	nombre_cols_efi <- c("Replicas", "NumMod", "Esq1", 
						"Esq2", "Esq3", "Esq4", "Esq5",
						 "Esq6", "Esq7", "Esq8")
	conteo_efi <- matrix(ncol=length(nombre_cols_efi),
						nrow = replicas)
	colnames(conteo_efi) <- nombre_cols_efi
	
	nombre_archivo_conteos <- paste(MODELO, "__", 
								CASO, "__N", N, "__resultados_conteo__",
								 ".xlsx", sep = "")
	nombre_archivo_efi <- paste(MODELO, "__", CASO, "__N", 
							N,"__resultados_eficiencia__", ".xlsx", sep = "")
	
	#Creando libros de excel
	wb_conteos <- createWorkbook()
	addWorksheet(wb_conteos, "Conteos")
	wb_eficiencia <- createWorkbook()
	addWorksheet(wb_eficiencia, "Eficiencia")
	
	for (replica in 1:replicas) {
		print(paste("Replica #", replica))
		num_m <- 0 #num de modelos
		#Conteos por replica
		conteo_replica <- matrix(0, nrow = 8,
						 ncol = length(nombre_cols))
		replica_vector <- rep(replica, each = esquemas)
		esquema_vector <- rep(1:esquemas)#vector de 1-8
		numMode_vector <- rep(limit_model, each = esquemas)
		#Matriz de datos iniciales
		matriz_inicial <- cbind(replica_vector,
						 esquema_vector,numMode_vector)
		conteo_replica[, 1:3] <- matriz_inicial
		#Eficiencia por replica
		conteo_efi_replica <- numeric(length(nombre_cols_efi))
		conteo_efi_replica[1] <- replica
		#Indices de incio y fin de replica
		fila_inicio <- (replica - 1) * N + 1
		fila_fin <- replica * N
		replica_data <- data_muestra[fila_inicio:fila_fin, ]
		R2_replica <- as.numeric(data_R2[replica, ])
		
		#Procesamiento de fila replica
		for (i in seq(1, ncol(replica_data), by = block)) {
			block_end <- min(i + block - 1, ncol(replica_data))
			block_caso <- replica_data[, i:block_end]
			R2_block <- as.numeric(R2_replica[i:500])
			Rmod <- 0 #Indice de R² a procesar
			
			#Procesamiendo de modelos replica
			for (j in seq(1, ncol(block_caso), by = cols_por_model)) {
				model_end <- min(j + cols_por_model - 1, ncol(block_caso))
				modeloActual <- block_caso[, j:model_end]
				R2_modelo <- R2_block[Rmod + 1]
				Rmod <- Rmod + 1
				resultadosInter <- EvalPrecisionModel(modeloActual,
											 alpha, nivConfianza, caso)
											 #Funcion propuesta
				print(R2_modelo)
				print(resultadosInter)
				
				#Procesando resultados del modelo por esquema
				for (numEsquema in 1:length(resultadosInter)) {
					resultados_esquema <- resultadosInter[[numEsquema]]
					intervalos_ganadores <- list()
					modelo_esquema_eficaz <- TRUE # Bandera de modelo eficaz en esquema
					
					#Procesando intervalos por esquema
					for (numIntervalo in 1:length(resultados_esquema)) {
						intervalo <- resultados_esquema[[numIntervalo]]
						# Verifica si el intervalo es NaN
						if (any(is.na(intervalo)) || length(intervalo) != 2) {
							modelo_esquema_eficaz <- FALSE
							warning(paste("Intervalo inválido en esquema",
											 numEsquema, "intervalo", numIntervalo, 
							"Intervalo:", paste(intervalo, collapse = ","), 
							"con R2_modelo:", R2_modelo))
							break
						} else {
							# Intervalo es válido, revisar si contiene el R2
							if (!is.na(R2_modelo)) {
								R2_intervalo <- ifelse(R2_modelo
												 >= intervalo[1] & R2_modelo <= intervalo[2], 1, 0)
								if (R2_intervalo == 1) {
									intervalos_ganadores[[length(intervalos_ganadores) + 1]]
									 <- list(Intervalo = numIntervalo, 
									 	Longitud = intervalo[2] - intervalo[1])
								}
							}
						}
					}#Fin procesado intervalos por esquema
					
					# Si ambos intervalos fueron calculados correctamente
					if (modelo_esquema_eficaz) {
						conteo_replica[numEsquema, 4] <-
							 conteo_replica[numEsquema, 4] + 1 #Modelo eficiente
						
						
						# Lógica para ganador único o empate
						if (length(intervalos_ganadores) == 1) {
							if (intervalos_ganadores[[1]]$Intervalo == 1){
						conteo_replica[numEsquema, 5] <- conteo_replica[numEsquema, 5] + 1 
							#Eficiencia
						conteo_replica[numEsquema, 7] <- conteo_replica[numEsquema, 7] + 1
							 #Ganador
							}
							if (intervalos_ganadores[[1]]$Intervalo == 2){
							conteo_replica[numEsquema, 6] <- conteo_replica[numEsquema, 6] + 1 
							#Eficiencia
							conteo_replica[numEsquema, 8] <- conteo_replica[numEsquema, 8] + 1
							 #Ganador
							}
							
						} else if (length(intervalos_ganadores) == 2) {
							#Conteo ambas contuvieron a R²
							conteo_replica[numEsquema, 5]<-conteo_replica[numEsquema, 5]+1
							conteo_replica[numEsquema, 6]<-conteo_replica[numEsquema, 6]+1
							conteo_efi_replica[numEsquema+2]<-conteo_efi_replica[
																			numEsquema+ 2]+1 
							#Entro en los dos
							mejor_intervalo <- intervalos_ganadores[[which.min(
												sapply(intervalos_ganadores, function(x) x$Longitud))]]
							if (mejor_intervalo$Intervalo == 1) 
								conteo_replica[numEsquema, 9]<-conteo_replica[numEsquema, 9]+1
							if (mejor_intervalo$Intervalo == 2) 
								conteo_replica[numEsquema, 10]<-conteo_replica[numEsquema, 10]+1
						}else{
						conteo_replica[numEsquema, 11] <- conteo_replica[numEsquema, 11]+1 
						#Sin ganadores
						}
					}
					
				}#Fin por esquemas
				
				#Segumiento de modelos
				num_m <- num_m + 1
				if (num_m %% 50 == 0) {
					print(paste("Procesados", m, "modelos de replica",
								 replica, "/",replicas))
				}
				if (num_m == limit_model) {
					break
				}
				
			}#Fin de modelos replica
		}#Fin de replica
		
		#Guardar resultados obtenidos de replica
		fila_inicio <- (replica - 1) * esquemas + 1
		fila_fin <- replica * esquemas
		conteos_totales[fila_inicio:fila_fin, ] <- conteo_replica
		conteo_efi_replica[2] <- num_m
		conteo_efi[replica, ] <- conteo_efi_replica
		
		#Guardar resultados en xlsx
		writeData(wb_conteos, "Conteos", conteos_totales, 
				startCol = 1, startRow = 1, rowNames = FALSE)
		writeData(wb_eficiencia, "Eficiencia", conteo_efi,
				 startCol = 1, startRow = 1, rowNames = FALSE)
		saveWorkbook(wb_conteos, nombre_archivo_conteos, overwrite = TRUE)
		saveWorkbook(wb_eficiencia, nombre_archivo_efi, overwrite = TRUE)
		
	}#Fin de replicas
	
	cat("Fin de cálculos")
}
\end{verbatim}


\textbf{Simulación de Modelos}
\begin{verbatim}

##SIMULACION DE MODELOS##
#N=numero de modelos a simular
#n=tamano de la muestra
#r=numero de replicas por tamano
#TipoPres=tipo de presicion del modelo;
# 1:Preciso; 2:Impreciso
#TipoSupues=tipo de supuesto que cumple 
#el modelo de regresion; 1:NVC; 2:NVD; 3:NNVC; 4:NNVD
#Los modelos son exactos, b0=0 y b1=1

SimMod=function(N,n,r,TipoPres,TipoSupues)
{library(writexl)
	cat("\n\n Simulacion de modelos de tamano",n,"\n")
	A=matrix(nrow=n*r,ncol=2*N)#Matriz de modelos
	B=matrix(nrow=r,ncol=N)#Matriz de R2
	b0=0
	b1=1
	for(j in 1:r)
	{s=n*(j-1)+1
		m=j*n
		for(i in 1:N)
		{if(TipoPres==1){R2=round(runif(1,min=0.8,max=0.99),4)}
			if(TipoPres==2){R2=round(runif(1,min=0.1,max=0.3),4)}
			B[j,i]=R2
			muz=round(runif(1,5,100),0)
			li=2*(i-1)+1
			ls=2*i
			if(TipoSupues==1){A[s:m,li:ls]=ModNVC(n,b0,b1,R2,muz)}
			if(TipoSupues==2){A[s:m,li:ls]=ModNVD(n,b0,b1,R2,muz)}
			if(TipoSupues==3){A[s:m,li:ls]=ModNNVC(n,b0,b1,R2,muz)}
			if(TipoSupues==4){A[s:m,li:ls]=ModNNVD(n,b0,b1,R2,muz)}
			if(i%%100==0){cat("\nYa se simularon",
						i,"modelos de la replica",j,"\n")}
		}
		cat("\n Listo la replica",j,"\n")
	}
	
	#Guarda los modelos
	A=as.data.frame(A)
	if(TipoPres==1 && TipoSupues==1){nomarch="EPNVC"}
	if(TipoPres==1 && TipoSupues==2){nomarch="EPNVD"}
	if(TipoPres==1 && TipoSupues==3){nomarch="EPNNVC"}
	if(TipoPres==1 && TipoSupues==4){nomarch="EPNNVD"}
	if(TipoPres==2 && TipoSupues==1){nomarch="EINVC"}
	if(TipoPres==2 && TipoSupues==2){nomarch="EINVD"}
	if(TipoPres==2 && TipoSupues==3){nomarch="EINNVC"}
	if(TipoPres==2 && TipoSupues==4){nomarch="EINNVD"}
	nomarch=paste(nomarch,n)
	nomarch=paste(nomarch,".xlsx")
	write_xlsx(A,path=nomarch)
	
	#Guarda las R2
	B=as.data.frame(B)
	if(TipoPres==1 && TipoSupues==1){nomarch2="R2EPNVC"}
	if(TipoPres==1 && TipoSupues==2){nomarch2="R2EPNVD"}
	if(TipoPres==1 && TipoSupues==3){nomarch2="R2EPNNVC"}
	if(TipoPres==1 && TipoSupues==4){nomarch2="R2EPNNVD"}
	if(TipoPres==2 && TipoSupues==1){nomarch2="R2EINVC"}
	if(TipoPres==2 && TipoSupues==2){nomarch2="R2EINVD"}
	if(TipoPres==2 && TipoSupues==3){nomarch2="R2EINNVC"}
	if(TipoPres==2 && TipoSupues==4){nomarch2="R2EINNVD"}
	nomarch2=paste(nomarch2,n)
	nomarch2=paste(nomarch2,".xlsx")
	write_xlsx(B,path=nomarch2)
	cat("\n\n Fin de la simulacion de modelos de tamano n=",n,"\n\n")
}#Termina la funcion SimMod
\end{verbatim}



\begin{verbatim}
#Genaración de las matrices de modelos y sus correspondiente R2 por cada tamano de muestra
#Modelos EPNVC por cada tamano de muestra
SimMod(N=500,n=10,r=5,TipoPres=1,TipoSupues=1)
SimMod(N=500,n=15,r=5,TipoPres=1,TipoSupues=1)
SimMod(N=500,n=20,r=5,TipoPres=1,TipoSupues=1)
SimMod(N=500,n=25,r=5,TipoPres=1,TipoSupues=1)
SimMod(N=500,n=30,r=5,TipoPres=1,TipoSupues=1)
SimMod(N=500,n=35,r=5,TipoPres=1,TipoSupues=1)

#Modelos EPNVD por cada tamano de muestra
SimMod(N=500,n=10,r=5,TipoPres=1,TipoSupues=2)
SimMod(N=500,n=15,r=5,TipoPres=1,TipoSupues=2)
SimMod(N=500,n=20,r=5,TipoPres=1,TipoSupues=2)
SimMod(N=500,n=25,r=5,TipoPres=1,TipoSupues=2)
SimMod(N=500,n=30,r=5,TipoPres=1,TipoSupues=2)
SimMod(N=500,n=35,r=5,TipoPres=1,TipoSupues=2)

#Modelos EPNNVC por cada tamano de muestra
SimMod(N=500,n=10,r=5,TipoPres=1,TipoSupues=3)
SimMod(N=500,n=15,r=5,TipoPres=1,TipoSupues=3)
SimMod(N=500,n=20,r=5,TipoPres=1,TipoSupues=3)
SimMod(N=500,n=25,r=5,TipoPres=1,TipoSupues=3)
SimMod(N=500,n=30,r=5,TipoPres=1,TipoSupues=3)
SimMod(N=500,n=35,r=5,TipoPres=1,TipoSupues=3)

#Modelos EPNNVD por cada tamano de muestra
SimMod(N=500,n=10,r=5,TipoPres=1,TipoSupues=4)
SimMod(N=500,n=15,r=5,TipoPres=1,TipoSupues=4)
SimMod(N=500,n=20,r=5,TipoPres=1,TipoSupues=4)
SimMod(N=500,n=25,r=5,TipoPres=1,TipoSupues=4)
SimMod(N=500,n=30,r=5,TipoPres=1,TipoSupues=4)
SimMod(N=500,n=35,r=5,TipoPres=1,TipoSupues=4)
################################################################################
################################################################################
#Modelos IPNVC por cada tamano de muestra
SimMod(N=500,n=10,r=5,TipoPres=2,TipoSupues=1)
SimMod(N=500,n=15,r=5,TipoPres=2,TipoSupues=1)
SimMod(N=500,n=20,r=5,TipoPres=2,TipoSupues=1)
SimMod(N=500,n=25,r=5,TipoPres=2,TipoSupues=1)
SimMod(N=500,n=30,r=5,TipoPres=2,TipoSupues=1)
SimMod(N=500,n=35,r=5,TipoPres=2,TipoSupues=1)

#Modelos IPNVD por cada tamano de muestra
SimMod(N=500,n=10,r=5,TipoPres=2,TipoSupues=2)
SimMod(N=500,n=15,r=5,TipoPres=2,TipoSupues=2)
SimMod(N=500,n=20,r=5,TipoPres=2,TipoSupues=2)
SimMod(N=500,n=25,r=5,TipoPres=2,TipoSupues=2)
SimMod(N=500,n=30,r=5,TipoPres=2,TipoSupues=2)
SimMod(N=500,n=35,r=5,TipoPres=2,TipoSupues=2)

#Modelos IPNNVC por cada tamano de muestra
SimMod(N=500,n=10,r=5,TipoPres=2,TipoSupues=3)
SimMod(N=500,n=15,r=5,TipoPres=2,TipoSupues=3)
SimMod(N=500,n=20,r=5,TipoPres=2,TipoSupues=3)
SimMod(N=500,n=25,r=5,TipoPres=2,TipoSupues=3)
SimMod(N=500,n=30,r=5,TipoPres=2,TipoSupues=3)
SimMod(N=500,n=35,r=5,TipoPres=2,TipoSupues=3)

#Modelos IPNNVD por cada tamano de muestra
SimMod(N=500,n=10,r=5,TipoPres=2,TipoSupues=4)
SimMod(N=500,n=15,r=5,TipoPres=2,TipoSupues=4)
SimMod(N=500,n=20,r=5,TipoPres=2,TipoSupues=4)
SimMod(N=500,n=25,r=5,TipoPres=2,TipoSupues=4)
SimMod(N=500,n=30,r=5,TipoPres=2,TipoSupues=4)
SimMod(N=500,n=35,r=5,TipoPres=2,TipoSupues=4)
\end{verbatim}