\documentclass[12pt]{article}
\usepackage{geometry}
\usepackage[utf8]{inputenc}
\usepackage[T1]{fontenc}
\usepackage{latexsym,amsmath,amssymb,amsfonts}
\usepackage[spanish]{babel} 
\usepackage{time}
\usepackage{setspace}          % Para manejar el espacio interlineal
\usepackage{fancyhdr}          % Para personalizar encabezados y pies de página
\usepackage{graphicx}
\usepackage{amsmath}
\usepackage{relsize}
\hyphenation{Mod-NVD}
\hyphenation{Datosz-y}
\usepackage{float}
\usepackage{placeins}
\usepackage[autostyle]{csquotes}
\usepackage[backend=biber,style=apa,sorting=nyt]{biblatex}
\addbibresource{referencias/Referencias.bib}
\spanishdecimal{.}
\usepackage{makecell} % para permitir saltos de línea en una celda

\usepackage{cellspace} % Para agregar padding vertical a las celdas

% Configuración de padding
\setlength{\tabcolsep}{5pt} % Padding horizontal
\setlength\cellspacetoplimit{10pt} % Padding vertical superior
\setlength\cellspacebottomlimit{10pt} % Padding vertical inferior

% Configuración de la sangría
\setlength{\parindent}{2ex}   % Sangría de 5 espacios (aproximadamente 5ex)

% Configuración del espacio interlineal
\onehalfspacing
\usepackage{enumitem} % Paquete para personalizar listas
 
\usepackage{array} % Para personalizar el estilo de las celdas
 
 % Configuración de padding
 \renewcommand{\arraystretch}{1.6} % Padding vertical (1.5 es el factor de espaciado)       
        

\geometry{
	paperwidth=21.59cm,
	paperheight=27.94cm,
	left=4.0cm,
	right=1.5cm,
	top=3cm,
	bottom=3cm,
}

\parindent = 0mm
 
\begin{document}
	\thispagestyle{empty}
	\hskip-2.15cm
	\begin{minipage}[c][1\totalheight][s]{3.5cm} 
		\begin{center}
			\includegraphics[height=2.5cm]{img/uady.png}\\[10pt]
		\end{center}
	\end{minipage}\begin{minipage}[c][1\totalheight][s]{12cm} 
		\begin{center}
			{\fontfamily{phv}\selectfont\Large\textbf {UNIVERSIDAD AUTÓNOMA DE YUCATÁN}}
			\vspace{0.3cm}
			\hrule height1.5pt
			\vspace{.3cm}
			{{\fontfamily{phv}\selectfont FACULTAD DE MATEMÁTICAS\\UNIDAD MULTIDISCIPLINARIA TIZIMÍN}}
		\end{center}
	\end{minipage}\begin{minipage}[c][1\totalheight][s]{3cm} 
		\begin{center}
			\includegraphics[height=2.5cm]{img/matem.png}\\[10pt]
		\end{center}
	\end{minipage}\quad
	\vspace{.01cm}
	
	\hskip -0.9cm
	\begin{minipage}[c][1\totalheight][s]{1cm}
		\begin{center}
			\hskip2pt\vrule width3.5pt height19.5cm\hskip1mm
			\vrule width0.5pt height19.5cm\\[10pt]
		\end{center}
	\end{minipage}\hspace{0.5cm}\begin{minipage}[c][1\totalheight][s]{14cm}
		\begin{center}
			\vspace{1.5cm}
			
			{\fontfamily{phv}\selectfont\Large\textbf{Una propuesta basada en estimadores Bootstrap robustos para la evaluación de la precisión de un modelo con la técnica de regresión lineal}}
			
			\vspace{1cm}
			{{\fontfamily{phv}\selectfont\large Tesis presentada por el:}}\\
			
			\vspace{0.5cm}
			
			{{\fontfamily{phv}\selectfont\large Br. Irving Geyler Cupul Uc}}\\
			
			\vspace{1.5cm}
			
			{\fontfamily{phv}\selectfont\large En su examen profesional en opción al título de:}
			
			\vspace{0.5cm}
			
			{\fontfamily{phv}\selectfont\large Licenciado en Ingeniería de Sofware}
			
			\vspace{1.2cm}
			
			{\fontfamily{phv}\selectfont\large Asesores:}
			
			\vspace{0.5cm}
			
			{\fontfamily{phv}\selectfont\large M.C. Luis Colorado Martínez\\M.C. Salvador Medina Peralta}
			
			\vspace{4cm}
			{{\fontfamily{phv}\selectfont Tizimín, Yucatán, Diciembre 2024}}
		\end{center}
	\end{minipage}
	\newpage
	\pagenumbering{gobble}  % Desactiva la numeración de páginas
	\input{2_agradecimientos-dedicacion.tex}
	\section*{Resumen}

En este trabajo se propone un método que permite evaluar la precisión de un modelo con la técnica de regresión lineal; y se basa en implementar diversos esquemas de remuestreo y estimar la precisión, a través de intervalos de confianza Bootstrap para el coeficiente de determinación $R^2$, del modelo de regresión entre los valores reales y predichos del modelo que se desea evaluar.\\

Tomando en cuenta el cumplimiento o no de los supuestos de normalidad y varianza constante, se consideraron cuatro escenarios posibles (NVC, NNVC, NVD y NNVD), como resultados de la combinación de ambos supuestos; y para estimar la distribución del coeficiente de determinación $R^2$, se implementaron ocho esquemas de remuestreo: el Bootstrap simple; el Wild Bootstrap robusto propuesto en \textcite{rana-2012}, ejecutado con los tres esquemas propuestos por \textcite{wu-1986} y con los dos esquemas propuestos por \textcite{liu-1988}; el Bootstrap de residuales balanceado y el Bootstrap Pareado Balanceado. Se proponen los intervalos percentiles y el Bca para estimar $R^2$ y para su cómputo se utilizan $B=1,000$ remuestras para cada uno de los esquemas Bootstrap.\\ 

Se realizó un estudio de simulación para comparar las eficiencias de los intervalos de confianza para cada tipo de supuesto con respecto a los diferentes esquemas Bootstrap, tamaños de muestra y tipo de modelo; para ello se simularon y evaluaron $60,000$ modelos Exactos-Precisos (EP) y $60,000$ modelos Exactos-Imprecisos (EI); para cada modelo se identificó la $R^2$ de origen utilizada para su simulación. Se simularon modelos de tamaños $n=10, 15, 20, 25, 30, 35$ para cada uno de los supuestos y tipo de modelo.\\

Se consideraron tres criterios principales para determinar las eficiencias de los intervalos para cada esquema Bootstrap, el primer criterio determina la eficiencia como el porcentaje de las veces en que el intervalo contiene a la $R^2$ de origen para los modelos EP simulados; y viceversa, para los modelos EI la eficiencia se determinó como el porcentaje de las veces en que el intervalo de confianza no contiene a la $R^2$ de origen. El segundo criterio determina la eficiencia como el porcentaje de las veces en que ambos intervalos contienen de manera simultánea a la $R^2$ de origen para los modelos EP y de manera viceversa cuando ambos no la contienen para los modelos EI; y el tercer criterio determina la eficiencia como el porcentaje de las veces en que uno de los intervalos es más estrecho que el otro cuando ambos intervalos contienen simultáneamente la $R^2$ de origen para los modelos EP y de manera viceversa uno de los intervalos es más estrecho que el otro cuando ambos no la contienen simultáneamente para los modelos EI.\\

Se analizaron los resultados del estudio de simulación a través de un ANOVA factorial y se determinó con al menos un 95\% que para los supuestos NVC, NNVC o NVD se utilice el ICB Percentil con el esquema de remuestreo Liu2 sin importar el tamaño de muestra. Y se determinó con al menos el 88.8\% que para el supuesto NNVD se utilice el ICB BCa con el esquema de remuestreo pareado balanceado; con la limitación de que para modelos EI con tamaños de muestra “pequeño” $n=10, 15, 20$ no se obtuvo un buen desempeño. \\


Con el propósito de que los resultados de este trabajo conformen una herramienta que permita evaluar la precisión de un modelo, se consideró como propuesta final: para los supuestos NVC, NNVC o NVD se utilice el ICB Percentil con el esquema de remuestreo Liu2 (NVC: residuales de regresión lineal simple, NNVC: residuales robustos sin ponderar y NVD: residuales robustos ponderados) y para el supuesto NNVD se utilice el ICB BCa con el esquema de remuestreo Pareado Balanceado. La propuesta final se implementó en el lenguaje R \parencite{R-2024}; y se ilustra con la evaluación de dos modelos que se ajustan a los diferentes escenarios contemplados en la propuesta y que corresponden a casos reales. Para la ganancia diaria de peso en ovinos, el modelo resultó ser de tipo NVC y preciso, coincidiendo con \textcite{balam-2012}, cabe señalar que usó ICB BCa con residuales balanceados y en este trabajo se usó ICB Percentil con el esquema de remuestreo Liu2. Para el volumen por parcela, el modelo resultó de tipo NNVD y preciso, coincidiendo con \textcite{balam-2012}, tanto en la decisión como en el esquema e ICB que utilizó.
	\newpage
	\tableofcontents
	\newpage
	\input{4_contenido-listas.tex}
	\newpage
	\pagenumbering{arabic}    % Usa números arábigos
	\setcounter{page}{1}      % Reinicia el contador a 1
	\section{Introducción}
Los modelos son representaciones matemáticas de mecanismos que rigen fenómenos naturales que no se reconocen, controlan o comprenden plenamente, y el proceso de modelación abarca varios pasos que comienzan con una declaración clara de los objetivos del modelo, supuestos sobre sus límites del modelo, la adecuación de los datos disponibles, el diseño de la estructura del modelo, la evaluación de las simulaciones y la aportación de la información para los procesos de recomendación y rediseño (Tedeschi, 2006).
\vspace{.5cm}
 
La validación de un modelo en predicción del sistema es la comparación por medio de algún método de las predicciones del modelo con los valores observados del sistema real para determinar su capacidad predictiva; en esta etapa del proceso de modelación matemática se evalúan la exactitud y precisión del modelo, la primera se refiere a la proximidad de las predicciones $( z )$ con los valores observados $( y )$, por ejemplo, sus diferencias $ ( d=y-z ) $ del cero y la segunda a la dispersión de los puntos $ (z, y) $; sin embargo, en presencia de exactitud la precisión se mide cuantificando la dispersión de dichos puntos respecto a una referencia, por ejemplo, la recta determinista $ y=x $, o bien, evaluar la varianza de las diferencias $ (\sigma_{D}^{2}) $ alrededor del cero $ (\mu_{D}=0) $ (Medina-Peralta et al., 2017).
\vspace{.5cm}
 
En la literatura se han expuesto diferentes enfoques y técnicas para validar modelos. Las técnicas de validación se pueden agrupar en cuatro categorías principales: evaluación subjetiva (involucra a un número de expertos en el campo de interés), técnicas visuales (gráficas comparativas), medidas de desviación (basadas en las diferencias entre valores observados y simulados) y pruebas de estadísticas (Mayer y Butler, 1993).
\vspace{.5cm}
 
Entre las técnicas inferenciales, una de las más utilizadas es la Regresión Lineal (RL) entre los observados del sistema real $ (y) $ en función de los predichos del modelo a evaluar $ (z) $, $ y_{i} = \beta_{0} + \beta_{1}z_{i} +\epsilon $  donde $ \epsilon_{i} \sim NI(0,\sigma^{2}) $ ; la exactitud se evalúa por medio de una $ F $ conjunta verificando si simultáneamente $ \beta_{0} $ y $ \beta_{1} $ son cero y uno respectivamente (Yang et al., 2004; Tedeschi, 2006; Montgomery et al., 2012); y la precisión se evalúa por medio del coeficiente de determinación $ R^{2} $ (Balam, 2012), mientras más cerca esté de uno el modelo es más preciso.
Adicionalmente, Zacarias (2023) desarrolló un método no paramétrico para evaluar la exactitud de un modelo con la técnica de regresión lineal cuando no se cumplen los supuestos de normalidad y/o varianza constante, basada en la construcción de una región de confianza Bootstrap para el vector de parámetros de regresión. De modo que si el vector  $ (\beta_{0},\beta_{1})=(0,1) $ está contenido en dicha región de confianza, se concluye el modelo es exacto en predicción del sistema. En Zacarias (2023), para garantizar que las estimaciones sean confiables y resistentes a las influencias de datos atípicos se utilizaron estimadores de regresión robustos y se implementó el Wild Bootstrap robusto propuesto en Rana et al. (2012) bajo tres esquemas de remuestreo propuestos por (Wu, 1986) y dos propuestos por (Liu, 1988).
\vspace{.5cm}

Febles (2014) propuso un método para simular modelos que implica la creación de una muestra pareada de valores observados y predichos $ (z_{i},y_{i}) $ basada en la media $ (\bar{z}) $ y los parámetros $ (\beta_{0}, \beta_{1}, R^{2}, SCE) $ de un modelo de regresión entre los observados y predichos. Este método simula modelos de cuatro tipos (exacto-preciso, exacto-impreciso, inexacto-preciso, inexacto-impreciso) y considera cuatro casos de supuestos (normalidad-varianza constante, no normalidad-varianza constante, normalidad-varianza no constante, no normalidad-varianza no constante) en función de la normalidad y la varianza en el modelo de regresión. En Zacarías (2023), se mejoraron los simuladores de Febles (2014) al seleccionar valores plausibles de $ SCE $ que generan modelos exacto-precisos independientemente del supuesto cumplido. Además, se determinó una constante para el ancho de una banda horizontal donde se distribuyen los valores predichos y residuales. Se reemplazó el estadístico de Kolmogorov-Smirnov por el de Lilliefors para las pruebas de normalidad, y se implementó la prueba de Breusch-Pagan cuando los residuales cumplen el supuesto de normalidad y la de White cuando los residuales no se ajustan a la distribución normal.
\vspace{.5cm}

Balam (2012), para medir la precisión de un modelo con regresión lineal, propone la construcción de un intervalo de confianza Bootstrap de residuales balanceado con sesgo corregido acelerado BCa, para los casos cuando se cumple el supuesto de varianza constante, independiente si se cumple o no el supuesto de normalidad; y para los casos cuando no se cumple el supuesto de varianza constante, independientemente si se cumple o no el de normalidad, se construye un intervalo de confianza Bootstrap pareado balanceado método percentil con sesgo corregido acelerado BCa.
\vspace{.5cm}

Para medir la precisión se puede considerar el Wild Bootstrap robusto propuesto en Rana et al. (2012) ya que este considera la utilización de estimadores robustos y otros esquemas de remuestreo diferente al Bootstrap simple.
En el presente trabajo se implementará una propuesta a través de intervalos de confianza para evaluar la precisión de un modelo con la técnica de regresión lineal, basada en estimadores robustos y los esquemas de remuestreo Bootstrap propuestos por Rana et al. (2012) e implementados en Zacarías (2023). Se realizará un estudio de simulación para evaluar la eficacia de esta propuesta y se implementará en el lenguaje R (R Core Team, 2024).

	\newpage
	\section{Objetivos}
\subsection{Objetivo general}
Determinar la precisión de un modelo con la técnica de regresión lineal por medio de intervalos de confianza basado en diferentes esquemas de remuestreo Bootstrap y medir sus eficiencias a través de un estudio de simulación.
\vspace{.5cm}
\subsection{Objetivos específicos}
\begin{enumerate}
\item Desarrollar la metodología para medir la precisión de un modelo con la técnica de regresión lineal por medio de intervalos de confianza basado en diferentes esquemas de remuestreo Bootstrap.
\item Determinar la precisión de un modelo cuando se cumplan los supuestos de normalidad y varianza constante.
\item Determinar la precisión de un modelo cuando no se cumplan los supuestos de normalidad y/o varianza constante.
\item Diseñar e implementar un estudio de simulación para evaluar la eficiencia de la metodología propuesta.
\item Simular modelos exactos-precisos (EP) y modelos exactos-imprecisos (EI) mediante la propuesta de \textcite{febles-2014} y \textcite{zacarias-2023}; cuando se cumplan o no los supuestos de normalidad e igualdad de varianzas.
\item Determinar la eficiencia de los esquemas Bootstrap propuestos para medir la precisión de un modelo.
\end{enumerate}

	\newpage
	\section{Marco Teórico}

\subsection{Validación de Modelos}
Los modelos son representaciones matemáticas de los mecanismos que rigen los fenómenos naturales (Tedeschi, 2006) o como una construcción matemática diseñada para estudiar un sistema del mundo real o fenómeno (Giordano et al., 1997).
\vspace{.5cm}

Medina-Peralta et al. (2017) indican que la validación de un modelo en la predicción del sistema implica la comparación, por medio de algún método, de las predicciones del modelo con los valores observados del sistema real para determinar su capacidad predictiva.
\vspace{.5cm}

Mayer y Butler (1993), clasifican los métodos de validación de modelos en Evaluación Subjetiva, Técnicas Visuales, Medidas de Desviación y Pruebas Estadísticas; también señalan que debido a las complejidades y tipos de datos, no existe una combinación establecida de técnicas de validación que sea aplicable en todas las áreas.
\vspace{.5cm}

Halachmi et al. (2004), menciona que la validación determina si el modelo matemático es una representación exacta del sistema real, y una forma de validación es comparando los datos reales con los predichos por el sistema.
\vspace{.5cm}

Para la validación de un modelo se evalúan la exactitud y la precisión; la primera se refiere a la proximidad de las predicciones $( z )$  con los valores observados $( y )$, por ejemplo, sus diferencias $ ( d=y-z ) $ del cero y la segunda a la dispersión de los puntos $ (z, y) $; además, en presencia de exactitud la precisión se mide cuantificando la dispersión de dichos puntos respecto a una referencia, por ejemplo, la recta determinística  $ y=x $, o bien, evaluar la varianza de las diferencias $ (\sigma_{D}^{2}) $ alrededor del cero $ (\mu_{D}=0) $ (Medina-Peralta et al., 2017).
\vspace{.5cm}

En la Figura \ref{fig:etiqueta} se ilustra la diferencia entre la exactitud y precisión de un modelo de simulación. El caso 1 es inexacto e impreciso, el caso 2 es inexacto y preciso, el caso 3 es exacto e impreciso y el caso 4 es exacto y preciso. En un modelo de predicción lo ideal es que cumpla el caso 4. 

\begin{figure}[ht]
	\centering
	\includegraphics[width=300px]{img/tadeshi_casos.png}
	\caption{Esquematización de Exactitud y Precisión. Fuente: Tedeschi (2006).}
	\label{fig:etiqueta}
\end{figure}

\subsection{Validación de Modelos con Regresión Lineal}


\subsection{Regresión Lineal Robusta}
\subsection{Wild Bootstrap}

\subsubsection{Algoritmo de remuestreo básico}

Se asume una muestra de $ x_{1}, x_{2}, ...,  x_{n}$ independiente e idénticamente distribuida.

\begin{enumerate}
		\item Se obtienen $B$ muestras de tamaño $n$ con reemplazo y con probabilidades iguales de la muestra original. La cardinalidad de este espacio muestra es $n^{n}$. Se denotan las muestras Bootstrap por $X^{*}_{1}, X^{*}_{2}, ..., X^{*}_{B}$
		
		\item 
\end{enumerate}



\subsubsection{Algoritmo de Remuestreo Balanceado}



\subsubsection{Algoritmo Bootstrap de Residuales (Rosalinda)}

Se asume que los $ \epsilon_{i} $ son independientes e idénticamente distribuidos. El algoritmo Bootstrap para generar muestras de $ R^{2} $ es el siguiente:

\begin{enumerate}
		\item Ajustar una regresión simple para el modelo $ y_{i} = \beta_{0} +\beta_{1}x_{i} + \epsilon_{i} $.
		\item Obtener los residuales $ \epsilon_{i} = y - \hat{y}   $
		, $i = 1,..., n $.
		\item  Remuestrear con probabilidades iguales la muestra $ e_{1},...,e_{n} $, para obtener $e^{*}_{11},...,e^{*}_{1n}$.
		\item Obtener $ y^{*}_{1i} = e^{*}_{1i} + \hat{y}_{1i}, i = 1, ..., n. $.
		\item Correr una regresión simple $ y^{*}_{1i} = \beta^{*}_{10} +\beta^{*}_{11}x_{i} + \epsilon^{*}_{1i} $ para obtener $ \hat{R}^{2*}_{1} $.
		\item Repetir los pasos 3 al 5, $B - 1$ veces para obtener las muestras: 
		\[
		\hat{R}^{2*}_{1} \hspace{.5cm} \hat{R}^{2*}_{2} \hspace{.5cm} \dots \hspace{.5cm} \hat{R}^{2*}_{B}
		\]
\end{enumerate}


\subsubsection{Algoritmo Bootstrap Pareado (Rosalinda)}

Supóngase que los datos surgieron de un estudio observacional donde ambas variables, $Y$ y $X$ son medidas de una colección de individuos seleccionados aleatoriamente. Supongamos que los $e_{i}$ en el modelo
$ y_{i} = \beta_{0} +\beta_{1}x_{i} + \epsilon_{i}$,    $i=1,2,..., n$ , no tienen varianza constante, lo que implica que no son idénticamente distribuidos (Givens y Hoeting, 2005; Montgomery et al., 2006).

\begin{enumerate}
	\item Considere la muestra $ w_{1} = (y_{1}, x_{1}),  w_{2} = (y_{2}, x_{2}), ..., w_{n} = (y_{n}, x_{n})$ como una muestra independiente e idénticamente distribuida donde la distribución es la conjunta $ F_{Y|X} $.
	\item  Tomar una muestra Bootstrap  $ w^{*}_{1}, w^{*}_{2},...,  w^{*}_{n} $ de $w_{1}, w_{2},...,  w_{n} $. Se obtienen la muestras  $y_{1}, y_{2},...,  y_{n} $ y  $x_{1}, x_{2},...,  x_{n} $.
	\item Correr una regresión lineal $ y^{*}_{i} = \beta_{i0} +\beta_{i1}x_{i}^{*} + \epsilon_{i} $.
	\item Estimar  $ \hat{R}^{2*}_{1} $.
	\item  Repetir los pasos 3 al 5, $B - 1$ veces para obtener las muestras: 
	\[
	\hat{R}^{2*}_{1} \hspace{.5cm} \hat{R}^{2*}_{2} \hspace{.5cm} \dots \hspace{.5cm} \hat{R}^{2*}_{B}
	\]
\end{enumerate}






\subsection{Simulación de Modelos}

\subsection{Diseño Factorial}
	\newpage
	\section{Metodología}
En esta sección presenta una propuesta para evaluar la precisión de modelos de regresión lineal mediante intervalos de confianza para el coeficiente de determinación \(R^2\). Se consideran modelos Exactos-Precisos (EP) y Exactos-Imprecisos (EI) bajo diferentes supuestos: Normalidad-Varianza Constante (NVC), No Normalidad-Varianza Constante (NNVC), Normalidad-Varianza Distinta (NVD) y No Normalidad-Varianza Distinta (NNVD). Se aplican esquemas de Bootstrap, incluyendo algoritmos de \textcite{wu-1986}, \textcite{liu-1988}, \textcite{zacarias-2023} y \textcite{balam-2012}, utilizando distintas técnicas de remuestreo y estimadores robustos según el cumplimiento de supuestos.

La metodología se valida mediante la simulación de 24,000 modelos distribuidos en 48 escenarios, cada uno con 500 modelos y 5 réplicas. Se consideran factores como el tamaño de muestra, precisión y supuestos. Se construyen intervalos Bootstrap (percentil y BCa) para determinar la precisión de \(R^2\), evaluando su eficacia mediante la frecuencia con la que los intervalos contienen el valor verdadero y analizando su ancho. Se realiza un análisis ANOVA factorial para identificar diferencias significativas entre los intervalos, complementado con pruebas de Tukey. Las simulaciones y análisis se implementan en el lenguaje R.


\subsection{Una Propuesta para Evaluar la Precisión de un Modelo}

Desarrollar la metodología para medir la precisión de un modelo con la técnica de regresión lineal por medio de intervalos de confianza basado en diferentes esquemas de remuestreo Bootstrap. 
\begin{figure}
	\centering 
	\includegraphics[width=0.70\linewidth]{img/metodo.png} 
	\caption{Algoritmo para los diferentes esquemas Bootstrap.}
\end{figure}
\

































\vspace{1.5cm}
\subsection{Precisión de modelos que cumplen el supuesto de normalidad y varianza constante}
Determinar la precisión de un modelo cuando se cumplan los supuestos de normalidad y varianza constante.
\vspace{1.5cm}





\subsection{Precisión de modelos que no cumplen el supuesto de normalidad y/o varianza constante}
Determinar la precisión de un modelo cuando no se cumplan los supuestos de normalidad y/o varianza constante.
\vspace{1.5cm}
	 
	 
\subsection{Estudio de simulación para la evaluación de la propuesta}
\vspace{1.5cm}

\subsubsection{Simulación de modelo}
Simular modelos exactos-precisos (EP) y modelos exactos-imprecisos (EI) mediante la propuesta de Febles (2014) y Zacarías (2023); cuando se cumplan o no los supuestos de normalidad e igualdad de varianzas.

%aqui va el diagrama
Diseñar e implementar un estudio de simulación para evaluar la eficacia de la metodología propuesta.
	 \vspace{1.5cm}
	 	 
	 	 
	 	 
	 	 
\subsection{Análisis estadísticos}
Para cada supuesto (NVC, NNVC, NVD, NNVD) se utilizó ANOVA en un arreglo factorial de tres factores seguido de la comparación múltiple de Tukey (Montgomery, 2017), para determinar el comportamiento de la eficacia de dos ICB en la evaluación de la precisión, bajo ocho esquemas de remuestreo, seis tamaños de muestra y dos tipos de modelo. Cabe señalar que, en cuatro de los ocho análisis de varianza realizados se eliminaron valores atípicos para el logro del cumplimiento de los supuestos del ANOVA.
Las pruebas estadísticas se consideraron significativas cuando  y se utilizó el paquete estadístico STATGRAPHICS Centurion 19 (Statgraphics, 2024).





	\newpage
	\section{Resultados}
Para los casos, exacto-preciso y exacto-impreciso; se registraron los porcentajes obtenidos después de la simulación de los 500 modelos correspondientes a la implementación de 5 replicas por cada caso y por cada tamaño de muestra. Para los casos inexactos-precisos, se consideraron.


\subsection{Eficiencia de los intervalos Bootstrap para el caso EP-NVC}
Se identifica una eficacia promedio de más del 95\% bajo el esquema robusto de Liu 2 al construir los intervalos Bootstrap percentil y BCa, obteniendo mayor precencia el intervalo Bootstrap percentil. Cuando se valida solo uno de los intervalos bajo sus esquemas, 


\begin{figure}[H] 
	\centering 
	\includegraphics[width=0.55\linewidth]{img/EP_NVC_Efic_Boots.png} 
	\caption{Eficiencia promedio de los intervalos Bootstrap por tamaño de muestra y esquema de remuestreos para el caso EP-NVC.} 
	\label{fig:EP_NVC_Boots}
\end{figure}

\FloatBarrier

\subsection{Eficiencia de los esquemas para el caso EP-NVC}
Con base en un $95\%$ de eficacia, se identifican en color rojo las e cacias
menores a este porcentaje con la nalidad de observar alguna relacion.
\begin{figure}[H] 
	\centering 
	\includegraphics[width=0.70\linewidth]{img/EP_NVC_Efic_Esq.png} 
	\caption{Eficiencia promedio de los esquemas por tamaño de muestra y esquema de remuestreos para el caso EP-NVC.} 
	\label{fig:EP_NVC_Esq}
\end{figure}

\FloatBarrier

	\newpage
	
Determinar la precisión de un modelo con la técnica de regresión lineal por medio de intervalos de confianza basado en diferentes esquemas de remuestreo Bootstrap y medir sus eficacias a través de un estudio de simulación.
\vspace{.5cm}
\subsection{Objetivos específicos}
\begin{enumerate}
	\item Desarrollar la metodología para medir la precisión de un modelo con la técnica de regresión lineal por medio de intervalos de confianza basado en diferentes esquemas de remuestreo Bootstrap.
	\item Determinar la precisión de un modelo cuando se cumplan los supuestos de normalidad y varianza constante.
	\item Determinar la precisión de un modelo cuando no se cumplan los supuestos de normalidad y/o varianza constante.
	\item Diseñar e implementar un estudio de simulación para evaluar la eficacia de la metodología propuesta.
	\item Simular modelos exactos-precisos (EP) y modelos exactos-imprecisos (EI) mediante la propuesta de \textcite{febles-2014} y \textcite{zacarias-2023}; cuando se cumplan o no los supuestos de normalidad e igualdad de varianzas.
	\item Determinar la eficacia de los esquemas Bootstrap propuestos para medir la precisión de un modelo.
\end{enumerate}



\section{Conclusiones}

%Las conclusiones son respuesta a mis objetivos

Cuando se tenga NVC, NNVC o NVD y se evalué la precisión, con base en al menos 95\% de eficiencia promedio, el ICB a utilizar sería Percentil con esquema de remuestreo Liu2.

Cuando se tenga NNVD y se evalué la precisión, el ICB a utilizar sería BCa con esquema de remuestreo ParBal. Cabe indicar que la eficiencia promedio fue de 88.8\% y sólo no identificó el tipo de modelo EI para tamaño de muestra “pequeño” n=10, 15, 20.
	\newpage
	\printbibliography
	\newpage
	\title{Anexo A. Resultados de Comparación de las Eficiencias de ICB}


Percentil-NVC

\begin{figure}[ht] 
	\centering 
	\includegraphics[width=0.95\linewidth]{img/ANOVA_Efic_ICB_Perc_NVC.png} 
	\caption{ANOVA para la eficiencia del ICB Percentil cuando se tiene NVC.} 
	\label{fig:ANOVA_Efic_ICB_Perc_NVC}
\end{figure}
\FloatBarrier


\begin{figure}[ht] 
	\centering 
	\includegraphics[width=0.76\linewidth]{img/CompEfic_PromICB_Perc_NVC.png} 
	\caption{Comparación de eficiencias promedio del ICB Percentil cuando se tiene NVC.} 
	\label{fig:CompEfic_PromICB_Perc_NVC}
\end{figure}
\FloatBarrier



BCa-NVC

\begin{figure}[ht] 
	\centering 
	\includegraphics[width=0.95\linewidth]{img/ANOVA_Efic_ICB_BCa_NVC.png} 
	\caption{ANOVA para la eficiencia del ICB BCa cuando se tiene NVC.} 
	\label{fig:ANOVA_Efic_ICB_BCa_NVC}
\end{figure}
\FloatBarrier


\begin{figure}[ht] 
	\centering 
	\includegraphics[width=0.76\linewidth]{img/CompEfic_PromICB_BCa_NVC.png} 
	\caption{Comparación de eficiencias promedio del ICB BCa cuando se tiene NVC.} 
	\label{fig:CompEfic_PromICB_BCa_NVC}
\end{figure}
\FloatBarrier



%%%%%%%%%%%%%%%%%%%%%%%%%%%%%%%%%%%%%%%%%%%5

Percentil-NNVC

\begin{figure}[ht] 
	\centering 
	\includegraphics[width=0.95\linewidth]{img/ANOVA_Efic_ICB_Perc_NNVC.png} 
	\caption{ANOVA para la eficiencia del ICB Percentil cuando se tiene NNVC.} 
	\label{fig:ANOVA_Efic_ICB_Perc_NNVC}
\end{figure}
\FloatBarrier


\begin{figure}[ht] 
	\centering 
	\includegraphics[width=0.76\linewidth]{img/CompEfic_PromICB_Perc_NNVC.png} 
	\caption{Comparación de eficiencias promedio del ICB Percentil cuando se tiene NNVC.} 
	\label{fig:CompEfic_PromICB_Perc_NNVC}
\end{figure}
\FloatBarrier



BCa-NNVC

\begin{figure}[ht] 
	\centering 
	\includegraphics[width=0.95\linewidth]{img/ANOVA_Efic_ICB_BCa_NNVC.png} 
	\caption{ANOVA para la eficiencia del ICB BCa cuando se tiene NNVC.} 
	\label{fig:ANOVA_Efic_ICB_BCa_NNVC}
\end{figure}
\FloatBarrier


\begin{figure}[ht] 
	\centering 
	\includegraphics[width=0.76\linewidth]{img/CompEfic_PromICB_BCa_NNVC.png} 
	\caption{Comparación de eficiencias promedio del ICB BCa cuando se tiene NNVC.} 
	\label{fig:CompEfic_PromICB_BCa_NNVC}
\end{figure}
\FloatBarrier



%%%%%%%%%%%%%%%%%%%%%%%%%%%%%%%%%%%%%%%%%%%5

Percentil-NVD

\begin{figure}[ht] 
	\centering 
	\includegraphics[width=0.95\linewidth]{img/ANOVA_Efic_ICB_Perc_NVD.png} 
	\caption{ANOVA para la eficiencia del ICB Percentil cuando se tiene NVD.} 
	\label{fig:ANOVA_Efic_ICB_Perc_NVD}
\end{figure}
\FloatBarrier


\begin{figure}[ht] 
	\centering 
	\includegraphics[width=0.76\linewidth]{img/CompEfic_PromICB_Perc_NVD.png} 
	\caption{Comparación de eficiencias promedio del ICB Percentil cuando se tiene NVD.} 
	\label{fig:CompEfic_PromICB_Perc_NVD}
\end{figure}
\FloatBarrier



BCa-NVD

\begin{figure}[ht] 
	\centering 
	\includegraphics[width=0.95\linewidth]{img/ANOVA_Efic_ICB_BCa_NNVC.png} 
	\caption{ANOVA para la eficiencia del ICB BCa cuando se tiene NNVC.} 
	\label{fig:ANOVA_Efic_ICB_BCa_NNVC}
\end{figure}
\FloatBarrier


\begin{figure}[ht] 
	\centering 
	\includegraphics[width=0.76\linewidth]{img/CompEfic_PromICB_BCa_NNVC.png} 
	\caption{Comparación de eficiencias promedio del ICB BCa cuando se tiene NNVC.} 
	\label{fig:CompEfic_PromICB_BCa_NNVC}
\end{figure}
\FloatBarrier





\subsection{Propuesta Final}

Con base en los resultados de los análisis estadísticos, cuando se tenga NVC, NNVC o NVD y se evalué la precisión el ICB a utilizar sería Percentil con esquema de remuestreo Liu2. Y cuando se tenga NNVD y se evalué la precisión, el ICB a utilizar sería BCa con esquema de remuestreo ParBal.


\subsubsection{Implementación}

\subsubsection{Aplicación}

  
	
\end{document}