
Determinar la precisión de un modelo con la técnica de regresión lineal por medio de intervalos de confianza basado en diferentes esquemas de remuestreo Bootstrap y medir sus eficacias a través de un estudio de simulación.
\vspace{.5cm}
\subsection{Objetivos específicos}
\begin{enumerate}
	\item Desarrollar la metodología para medir la precisión de un modelo con la técnica de regresión lineal por medio de intervalos de confianza basado en diferentes esquemas de remuestreo Bootstrap.
	\item Determinar la precisión de un modelo cuando se cumplan los supuestos de normalidad y varianza constante.
	\item Determinar la precisión de un modelo cuando no se cumplan los supuestos de normalidad y/o varianza constante.
	\item Diseñar e implementar un estudio de simulación para evaluar la eficacia de la metodología propuesta.
	\item Simular modelos exactos-precisos (EP) y modelos exactos-imprecisos (EI) mediante la propuesta de \textcite{febles-2014} y \textcite{zacarias-2023}; cuando se cumplan o no los supuestos de normalidad e igualdad de varianzas.
	\item Determinar la eficacia de los esquemas Bootstrap propuestos para medir la precisión de un modelo.
\end{enumerate}



\section{Conclusiones}

%Las conclusiones son respuesta a mis objetivos

Cuando se tenga NVC, NNVC o NVD y se evalué la precisión, con base en al menos 95\% de eficiencia promedio, el ICB a utilizar sería Percentil con esquema de remuestreo Liu2.

Cuando se tenga NNVD y se evalué la precisión, el ICB a utilizar sería BCa con esquema de remuestreo ParBal. Cabe indicar que la eficiencia promedio fue de 88.8\% y sólo no identificó el tipo de modelo EI para tamaño de muestra “pequeño” n=10, 15, 20.